\documentclass[12pt]{article}

\usepackage{amsmath, amssymb, amsthm}
\usepackage{enumerate}
\usepackage{multicol}
\usepackage{listings}
\usepackage{changepage}
\usepackage{tikz}
\usepackage{pgfplots, pgfplotstable}
\usepackage{caption}

% Remove paragraph indents
\setlength{\parindent}{0in}

\title{CS475: Project 3}
\author{Trevor Bramwell}
\date{\today}

\begin{document}
\maketitle

\section*{False Sharing}

This project used OpenMP to show how false-sharing can affect the
performance of parallel applications. These insights were gained by
writing a program that performed computations on a C struct with
variable end-padding.

The program was compiled using \emph{gcc} (without compiler
optimizations) and ran on the \emph{rabbit.engr.oregonstate.edu} Xeon
processor with 32 CPUs (not the Xeon Phi). It used the following combination
of parameters:

\begin{itemize}
    \item Padding Size (in integers):
        \begin{multicols}{2}
            \begin{itemize}
                \item 0
                \item 1
                \item 2
                \item 3
                \item 4
                \item 5
                \item 6
                \item 7
                \item 8
                \item 9
                \item 10
                \item 11
                \item 12
                \item 14
                \item 15
                \item 16
            \end{itemize}
        \end{multicols}
    \item Threads:
        \begin{multicols}{2}
            \begin{itemize}
                \item 1
                \item 2
                \item 4
                \item 8
                \item 16
            \end{itemize}
        \end{multicols}
\end{itemize}

Along with the combinations of the above parameters, the program was
also ran with a private variable maintained by each thread for
accumulation. This put a stop to the false-sharing, and showed a more
consistent improvement in performance when more threads were added.


\section*{Graph}

\pgfplotsset{width=5in}

\begin{tikzpicture}
    \begin{semilogyaxis}[
        ymin=0,
        xtick={0,1,2,3,4,5,6,7,8,9,10,11,12,13,14,15,16},
        title=Mega Increments per Second vs. Padding,
        xlabel=Padding,
        ylabel=Mega Increments per Second,
        log basis y={2},
        legend style={
            legend pos=outer north east
        }
    ]
    \addplot+[smooth] table[x=NUM,y=PERF] {../data/padding1.dat};
    \addplot+[smooth] table[x=NUM,y=PERF] {../data/padding2.dat};
    \addplot+[smooth,green!70!black] table[x=NUM,y=PERF] {../data/padding4.dat};
    \addplot+[smooth,magenta] table[x=NUM,y=PERF] {../data/padding8.dat};
    \addplot+[smooth,brown] table[x=NUM,y=PERF] {../data/padding16.dat};
    \addplot[mark=none,thick,blue,domain=0:16] {35};
    \addplot[mark=none,thick,red,domain=0:16] {64};
    \addplot[mark=none,thick,green!70!black,domain=0:16] {112};
    \legend{1, 2, 4, 8, 16, 1-p, 2-p, {4,8,16-p}}
    \addlegendimage{empty legend}
    \addlegendentry{Threads \hfill}
    \end{semilogyaxis}
\end{tikzpicture}

\section*{Patterns}

\paragraph{Private Variables}
The lines 1-p, 2-p, and 4,8,16-p are the average performance per thread
using private variables. They are baselines for the best
average-performance for those numbers of threads.

\paragraph{One Thread}
With a single thread there is no performance gain as it is being ran on
a single CPU and can't take advantage of spatial coherence.

\paragraph{Two Threads}
With two threads, performance stays constant after a padding of 3-4. I
believe this is because the number of cache lines is greater than 2, and
being limited to two threads, the program can't take advantage of the
other cache lines.

\paragraph{4,8,16 Threads}
The following is an explanation of the performance of the program while
padding is increased each step of the way.

\begin{description}
    \item[0:] Without any padding, performance was abysmal (worse than a
        single thread) even with 16 threads. This is the most important
        data point since it helps characterize the rest.
    \item[1:] There is the slightest increase in performance.
    \item[2:] Performance is continuing to increase, though the greatest
        comes with two threads. Remember that the Xeon has 2 CPU
        sockets. This same performance can be seen in the 4,8, and 16
        threads, but at a lower gain because of the cache-miss overhead.
    \item[3-6:] Performance peaks and begins to stabilize.
    \item[7:] There is a distinct drop in performance as false-sharing
        has started. With 4 or more threads each cache line is reloaded
        with the same data multiple times.
    \item[8-12:] There is little change in performance.
    \item[13-16:] The cache lines are now all being taken advantage of,
        false-sharing has stopped, and the best performance has been
        reached.
\end{description}

There is difference in the amount of padding needed before false-sharing
stops, between this graph and the one in the notes. The performance
increases at 13, while the notes show an increase at 15. This is because
of the Xeon most likely has a slightly smaller 4-Way L1 cache than the
machine the notes used.

\twocolumn

\section*{Tables}

\begin{figure}[h]
\pgfplotstabletypeset[string type]{../data/padding1.dat}
\captionof{table}{Padding - 1 Thread}
\end{figure}

\begin{figure}[h]
\pgfplotstabletypeset[string type]{../data/padding2.dat}
\captionof{table}{Padding - 2 Threads}
\end{figure}

\begin{figure}[h]
\pgfplotstabletypeset[string type]{../data/padding4.dat}
\captionof{table}{Padding - 4 Threads}
\end{figure}

\begin{figure}[h]
\pgfplotstabletypeset[string type]{../data/padding8.dat}
\captionof{table}{Padding - 8 Threads}
\end{figure}

\begin{figure}[h]
\pgfplotstabletypeset[string type]{../data/padding16.dat}
\captionof{table}{Padding - 16 Threads}
\end{figure}

\twocolumn

\section*{Tables - Private Sum}

\begin{figure}[h]
\pgfplotstabletypeset[string type]{../data/padding1-private.dat}
\captionof{table}{Padding - 1 Thread - Private Sum}
\end{figure}

\begin{figure}[h]
\pgfplotstabletypeset[string type]{../data/padding2-private.dat}
\captionof{table}{Padding - 2 Threads - Private Sum}
\end{figure}

\begin{figure}[h]
\pgfplotstabletypeset[string type]{../data/padding4-private.dat}
\captionof{table}{Padding - 4 Threads - Private Sum}
\end{figure}

\begin{figure}[h]
\pgfplotstabletypeset[string type]{../data/padding8-private.dat}
\captionof{table}{Padding - 8 Threads - Private Sum}
\end{figure}

\begin{figure}[h]
\pgfplotstabletypeset[string type]{../data/padding16-private.dat}
\captionof{table}{Padding - 16 Threads - Private Sum}
\end{figure}

\end{document}
