\documentclass[12pt]{article}

\usepackage{amsmath, amssymb, amsthm}
\usepackage{enumerate}
\usepackage{multicol}
\usepackage{listings}
\usepackage{changepage}
\usepackage{tikz}
\usepackage{pgfplots, pgfplotstable}

\title{CS475: Project 1}
\author{Trevor Bramwell}
\date{\today}

\begin{document}
\maketitle

\section*{Numerical Integration}

This assignment involved computing the volume under a Bezier surface using
the OpenMP library. The program was compiled using \emph{gcc} and ran 10
times on \emph{rabbit.engr.oregonstate.edu}, a 32 CPU machine, using the
following combination of parameters:
\\
\\
Subdivisions:
\begin{multicols}{2}
\begin{itemize}
    \item 1000
    \item 2000
    \item 4000
    \item 8000
    \item 16000
    \item 32000
    \item 46340
\end{itemize}
\end{multicols}

\noindent Threads:
\begin{multicols}{2}
\begin{itemize}
    \item 1
    \item 2
    \item 4
    \item 8
    \item 16
    \item 32
    \item 64
    \item 128
\end{itemize}
\end{multicols}

\noindent Each run output the number of subdivision of the problem, the
number of threads used, Mega Heights Computer Per Second (MHCPs), and
total calculated volume. After all 10 runs the average MHCPs was
computed for each group of subdivisions and threads. These results can
be seen in the Tables section at the end of this report. A LibreOffice
Calc document containing all the data is included in the \emph{data}
directory of the project.

\noindent The results after 10 runs agree on the volume under the Bezier
surface equalling: ${\bf14.0625}$.

\section*{Graphs}

\pgfplotsset{width=5in}

\begin{tikzpicture}
    \begin{loglogaxis}[
        ymin=0,
        title=Mega Heights Computed Per Second(MHCPs) vs. Subdivisions,
        xlabel=Subdivisions,
        ylabel=MHCPs,
        log basis x={2},
        log basis y={2},
        legend style={
            legend pos=outer north east
        }
    ]
    \addplot+[smooth] table[x=NUMS,y=MHCPS] {tables/tab1-1.dat};
    \addplot+[smooth] table[x=NUMS,y=MHCPS] {tables/tab1-2.dat};
    \addplot+[smooth] table[x=NUMS,y=MHCPS] {tables/tab1-4.dat};
    \addplot+[smooth] table[x=NUMS,y=MHCPS] {tables/tab1-8.dat};
    \addplot+[smooth] table[x=NUMS,y=MHCPS] {tables/tab1-16.dat};
    \addplot+[smooth] table[x=NUMS,y=MHCPS] {tables/tab1-32.dat};
    \addplot+[smooth] table[x=NUMS,y=MHCPS] {tables/tab1-64.dat};
    \addplot+[smooth] table[x=NUMS,y=MHCPS] {tables/tab1-128.dat};
    \legend{1,2,4,8,16,32,64,128}
    \addlegendimage{empty legend}
    \addlegendentry{Threads \hfill}
    \end{loglogaxis}
\end{tikzpicture}\\

\noindent It can be seen from this graph that adding more threads consistently
increased the MHCPs. Starting at 32 threads, MHCPs dropped before
increasing at a faster rate than lower numbers of threads. This is not
very surprising because rabbit only has 32 CPUs, and is most likely due
to the overhead incurred through process scheduling.\\
\\
\noindent Another interesting point is that MHCPs didn't increase until the
subdivisions reached 8192. Given the number of CPUs on rabbit is 32, we
can see: $8192/32 = 512 = (2^{13})$. I'm unsure as to why 512 is a
special number in this instance.\\

\begin{tikzpicture}
    \begin{semilogxaxis}[
        ymin=0,
        xtick={0,2,4,8,16,32,64,128,256},
        title=Mega Heights Computed Per Second(MHCPs) vs. Threads,
        xlabel=Threads,
        ylabel=MHCPs,
        log basis x={2},
        legend style={
            legend pos=outer north east
        }
    ]
    \addplot+[smooth] table[x=NUMT,y=MHCPS] {tables/tab2-1.dat};
    \addplot+[smooth] table[x=NUMT,y=MHCPS] {tables/tab2-2.dat};
    \addplot+[smooth] table[x=NUMT,y=MHCPS] {tables/tab2-4.dat};
    \addplot+[smooth] table[x=NUMT,y=MHCPS] {tables/tab2-8.dat};
    \addplot+[smooth] table[x=NUMT,y=MHCPS] {tables/tab2-16.dat};
    \addplot+[smooth] table[x=NUMT,y=MHCPS] {tables/tab2-32.dat};
    \addplot+[smooth] table[x=NUMT,y=MHCPS] {tables/tab2-64.dat};
    \legend{1000,2000,4000,8000,16000,32000,46430}
    \addlegendimage{empty legend}
    \addlegendentry{Subdivision \hfill}
    \end{semilogxaxis}
\end{tikzpicture}\\

\noindent In this graph of MHCPs vs Threads, it is clear to see the
scheduling overhead reduce MHCPs when threads surpass the number of CPUs
available.

\section*{Speedup}

Since these computations were run on a 32 CPU machine, the speedup for
each can be determined by dividing the MHCPs by 32. Using this speedup
we can determine the fraction of the program that is parallelizable.
The largest MHCPs (554.28) was computed using 64 threads and 46,340
subdivisions.

Using Amdahl's Inverse Law, we have:

\begin{align*}
    F_{parallel} &= \frac{n}{n-1}\left(1-\frac{1}{Speedup}\right)\\
                 &= \frac{32}{31}\left(1-\frac{1}{(554.28/32)}\right)\\
                 &= 0.97266
\end{align*}

Given that 0.97266 (97.26\%) of the program is parallelizable,
only 0.02734 (2.7\%) is sequential. Using Amdahl's Law, we can see the
max speedup this program could ever achive is:

\begin{align*}
    maxSpeedup &= \frac{1}{1-F_{parallel}}\\
               &= \frac{1}{1-0.97266}\\
               &= 36.57644
\end{align*}


\section*{Tables}

\pgfplotstabletypeset{tables/tab1-1.dat}|
\pgfplotstabletypeset{tables/tab1-2.dat}|
\pgfplotstabletypeset{tables/tab1-4.dat}
\pgfplotstabletypeset{tables/tab1-8.dat}|
\pgfplotstabletypeset{tables/tab1-16.dat}|
\pgfplotstabletypeset{tables/tab1-32.dat}
\pgfplotstabletypeset{tables/tab1-64.dat}|
\pgfplotstabletypeset{tables/tab1-128.dat}


\end{document}
