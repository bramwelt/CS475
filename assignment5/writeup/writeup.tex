\documentclass[12pt]{article}

\usepackage{amsmath, amssymb, amsthm}
\usepackage{enumerate}
\usepackage{multicol}
\usepackage{listings}
\usepackage{listings}
\usepackage{changepage}
\usepackage{tikz}
\usepackage{pgfplots, pgfplotstable}
\usepackage{caption}

\usepgfplotslibrary{dateplot}
\usepgfplotslibrary{colormaps}

\lstset{
    basicstyle=\tiny\ttfamily
}

% Remove paragraph indents
\setlength{\parindent}{0in}

\title{CS475: Project 5}
\author{Trevor Bramwell}
\date{\today}

\begin{document}
\maketitle

\section*{OpenCL Array Multiply, Multiply-Add, and Multiply-Reduce}

This assignment was compiled on my laptop, a Thinkpad W540 with 8 CPUs
(1 sockets, 4 cores, 2 threads per core) and a Nvidia Quatro K1100M.

I had three major problems getting my program to compile and run:
\begin{itemize}
    \item Proprietary Graphics
        Though the Nvidia Quatro K1100M supports OpenCL, from my
        experience the open source drivers do not. There may be CUDA
        drivers, or an OpenCL SDK in a seperate package, but they are
        bundled with the open source driver, nor is there any solid
        information online regarding getting OpenCL working with the
        open source drivers.
    \item Link Flags
        My C programming knowledge is limited, and a lot of time was
        spent trying to gcc to link the program against the OpenCL
        library. This most likely worked from the beginning, and my
        issues was merely having the link flags before the source file,
        instead of after.
    \item Multiple GPUs (Bumblebee)
        The W540 Thinkpad actually contains two GPUs: an integrated
        Intel graphics processor, and the Quatro K1100M. The printinfo
        command showed both cards, with only the Quatro supporting
        OpenCL. I had to change the first.cpp file to query for both
        devices, and use the second (Quatro) one.

After these hurdles I was able to compile, run, and recieved data from
the program. 

\section*{Graph}

\pgfplotsset{width=5in}
\pgfplotstableread{../data/graindeer.dat}\graindeer

\begin{tikzpicture}
    \begin{axis}[
        ymin=-10,
        title=Growth vs. Time,
        xlabel=Time,
        ylabel=Growth,
        legend style={
            legend pos=outer north east
        }
    ]
    \addplot+[smooth, mark=none] table[y=TEMP(C), x=STEP] {\graindeer};
    \addplot+[smooth, mark=none] table[y=PRECIP(cm), x=STEP] {\graindeer};
    \addplot+[smooth, mark=none,color=green!60!black] table[y=GRAIN(cm), x=STEP] {\graindeer};
    \addplot+[smooth, mark=none] table[y=DEER, x=STEP] {\graindeer};
    \addplot+[smooth, mark=none, color=orange, thick] table[y=FIRE, x=STEP] {\graindeer};
    \legend{Temperature, Precipitation, Grain, Deer, Fire}
    \addlegendimage{empty legend}
    \addlegendentry{Simultation \hfill}
    \end{axis}
\end{tikzpicture}

\section*{Patterns}

As the grain increased, more deer were born and ate more. Each time a
wildfire broke out the grain was reduced significantly, and as a result
the deer started dying.

\section*{Tables}

\begin{figure*}[b]
\lstinputlisting{../data/graindeer.dat}
\captionof{table}{Grain Deer Simulation}
\end{figure*}


\end{document}
