\documentclass[12pt]{article}

\usepackage{amsmath, amssymb, amsthm}
\usepackage{enumerate}
\usepackage{multicol}
\usepackage{listings}
\usepackage{listings}
\usepackage{changepage}
\usepackage{tikz}
\usepackage{pgfplots, pgfplotstable}
\usepackage{caption}

\usepgfplotslibrary{dateplot}
\usepgfplotslibrary{colormaps}

\lstset{
    basicstyle=\small\ttfamily
}

% Remove paragraph indents
\setlength{\parindent}{0in}

\title{CS475: Project 5}
\author{Trevor Bramwell}
\date{\today}

\begin{document}
\maketitle

\section*{OpenCL Array Multiply, Multiply-Add, and Multiply-Reduce}

This assignment was compiled on my laptop, a Thinkpad W540 with 8 CPUs
(1 sockets, 4 cores, 2 threads per core) and a Nvidia Quadro K1100M.

I had three major problems getting my program to compile and run:
\begin{itemize}
    \item Proprietary Graphics
        Though the Nvidia Quadro K1100M supports OpenCL, from my
        experience the open source drivers do not. There may be CUDA
        drivers, or an OpenCL SDK in a seperate package, but they are
        bundled with the open source driver, nor is there any solid
        information online regarding getting OpenCL working with the
        open source drivers.
    \item Link Flags
        My C programming knowledge is limited, and a lot of time was
        spent trying to gcc to link the program against the OpenCL
        library. This most likely worked from the beginning, and my
        issues was merely having the link flags before the source file,
        instead of after.
    \item Multiple GPUs (Bumblebee)
        The W540 Thinkpad actually contains two GPUs: an integrated
        Intel graphics processor, and the Quadro K1100M. The printinfo
        command showed both cards, with only the Quadro supporting
        OpenCL. I had to change the first.cpp file to query for both
        devices, and use the second (Quadro) one.
\end{itemize}

After these hurdles I was able to compile, run, and received data from
the program. I did not have time to look into Multiply-reductions, nor
was I able to fit everything into two graphs. The graph issue is due to
me having too much data.

\section*{Graph: Multiply (Local)}

\pgfplotsset{width=5in}

\begin{tikzpicture}
    \begin{loglogaxis}[
        ymin=0,
        log basis x={2},
        title=Performance vs. Local Work Size,
        ylabel=GigaMultsPerSecond,
        xlabel=Local Work Size,
        legend style={
            legend pos=outer north east
        }
    ]
    \addplot+[smooth] table[y=GMPS, x=LOCAL_SIZE] {../data/vector-mult-1.dat};
    \addplot+[smooth] table[y=GMPS, x=LOCAL_SIZE] {../data/vector-mult-2.dat};
    \addplot+[smooth] table[y=GMPS, x=LOCAL_SIZE] {../data/vector-mult-3.dat};
    \addplot+[smooth] table[y=GMPS, x=LOCAL_SIZE] {../data/vector-mult-4.dat};
    \addplot+[smooth] table[y=GMPS, x=LOCAL_SIZE] {../data/vector-mult-5.dat};
    \addplot+[smooth] table[y=GMPS, x=LOCAL_SIZE] {../data/vector-mult-6.dat};
    \addplot+[smooth] table[y=GMPS, x=LOCAL_SIZE] {../data/vector-mult-7.dat};
    \addplot+[smooth] table[y=GMPS, x=LOCAL_SIZE] {../data/vector-mult-8.dat};
    \addplot+[smooth] table[y=GMPS, x=LOCAL_SIZE] {../data/vector-mult-9.dat};
    \addplot+[smooth] table[y=GMPS, x=LOCAL_SIZE] {../data/vector-mult-10.dat};
    \addplot+[smooth] table[y=GMPS, x=LOCAL_SIZE] {../data/vector-mult-11.dat};
    \addplot+[smooth] table[y=GMPS, x=LOCAL_SIZE] {../data/vector-mult-12.dat};
    \addplot+[smooth] table[y=GMPS, x=LOCAL_SIZE] {../data/vector-mult-13.dat};
    \addplot+[smooth] table[y=GMPS, x=LOCAL_SIZE] {../data/vector-mult-14.dat};
    \legend{1024, 2048, 4096, 8192, 16384, 32768, 65536, 131072, 262144, 524288, 1048576, 2097152, 4194304, 8388608}
    \addlegendimage{empty legend}
    \addlegendentry{Local Work Size \hfill}
    \end{loglogaxis}
\end{tikzpicture}

\section*{Graph: Multiply-Add (Local)}

\begin{tikzpicture}
    \begin{loglogaxis}[
        ymin=0,
        log basis x={2},
        title=Performance vs. Local Work Size,
        ylabel=GigaMult-AddsPerSecond,
        xlabel=Local Work Size,
        legend style={
            legend pos=outer north east
        }
    ]
    \addplot+[smooth] table[y=GMPS, x=LOCAL_SIZE] {../data/vector-mult-add-1.dat};
    \addplot+[smooth] table[y=GMPS, x=LOCAL_SIZE] {../data/vector-mult-add-2.dat};
    \addplot+[smooth] table[y=GMPS, x=LOCAL_SIZE] {../data/vector-mult-add-3.dat};
    \addplot+[smooth] table[y=GMPS, x=LOCAL_SIZE] {../data/vector-mult-add-4.dat};
    \addplot+[smooth] table[y=GMPS, x=LOCAL_SIZE] {../data/vector-mult-add-5.dat};
    \addplot+[smooth] table[y=GMPS, x=LOCAL_SIZE] {../data/vector-mult-add-6.dat};
    \addplot+[smooth] table[y=GMPS, x=LOCAL_SIZE] {../data/vector-mult-add-7.dat};
    \addplot+[smooth] table[y=GMPS, x=LOCAL_SIZE] {../data/vector-mult-add-8.dat};
    \addplot+[smooth] table[y=GMPS, x=LOCAL_SIZE] {../data/vector-mult-add-9.dat};
    \addplot+[smooth] table[y=GMPS, x=LOCAL_SIZE] {../data/vector-mult-add-10.dat};
    \addplot+[smooth] table[y=GMPS, x=LOCAL_SIZE] {../data/vector-mult-add-11.dat};
    \addplot+[smooth] table[y=GMPS, x=LOCAL_SIZE] {../data/vector-mult-add-12.dat};
    \addplot+[smooth] table[y=GMPS, x=LOCAL_SIZE] {../data/vector-mult-add-13.dat};
    \addplot+[smooth] table[y=GMPS, x=LOCAL_SIZE] {../data/vector-mult-add-14.dat};
    \legend{1024, 2048, 4096, 8192, 16384, 32768, 65536, 131072, 262144, 524288, 1048576, 2097152, 4194304, 8388608}
    \addlegendimage{empty legend}
    \addlegendentry{Local Work Size \hfill}
    \end{loglogaxis}
\end{tikzpicture}

There is very little difference between these two graphs when varying the local
work group size. This makes sense as the OpenCL operations being performed are
independent of each other, and do not share any data. I have no clues as to why
the outliers at $2^8$ exist only for the non-multiply-and-add.

\section*{Graphs: Multiply (Global)}

\begin{tikzpicture}
    \begin{loglogaxis}[
        ymin=0,
        log basis x={2},
        title=Performance vs. Global Work Size,
        ylabel=GigaMultsPerSecond,
        xlabel=Global Work Size,
        legend style={
            legend pos=outer north east
        }
    ]
    \addplot+[smooth] table[y=GMPS, x=NUM_WORK_GROUPS] {../data/vector-mult-group-1.dat};
    \addplot+[smooth] table[y=GMPS, x=NUM_WORK_GROUPS] {../data/vector-mult-group-2.dat};
    \addplot+[smooth] table[y=GMPS, x=NUM_WORK_GROUPS] {../data/vector-mult-group-4.dat};
    \addplot+[smooth] table[y=GMPS, x=NUM_WORK_GROUPS] {../data/vector-mult-group-8.dat};
    %\addplot+[smooth] table[y=GMPS, x=NUM_WORK_GROUPS] {../data/vector-mult-group-16.dat};
    \addplot+[smooth] table[y=GMPS, x=NUM_WORK_GROUPS] {../data/vector-mult-group-32.dat};
    \addplot+[smooth] table[y=GMPS, x=NUM_WORK_GROUPS] {../data/vector-mult-group-64.dat};
    \addplot+[smooth] table[y=GMPS, x=NUM_WORK_GROUPS] {../data/vector-mult-group-128.dat};
    \addplot+[smooth] table[y=GMPS, x=NUM_WORK_GROUPS] {../data/vector-mult-group-256.dat};
    \addplot+[smooth] table[y=GMPS, x=NUM_WORK_GROUPS] {../data/vector-mult-group-512.dat};
    \addplot+[smooth] table[y=GMPS, x=NUM_WORK_GROUPS] {../data/vector-mult-group-1024.dat};
    \legend{1, 2, 4, 8, 32, 64, 128, 256, 512, 1024}
    \addlegendimage{empty legend}
    \addlegendentry{Global Size \hfill}
    \end{loglogaxis}
\end{tikzpicture}

\section*{Graphs: Multiply-Add (Global)}

\begin{tikzpicture}
    \begin{loglogaxis}[
        ymin=0,
        log basis x={2},
        title=Performance vs. Global Work Size,
        ylabel=GigaMult-AddsPerSecond,
        xlabel=Global Work Size,
        legend style={
            legend pos=outer north east
        }
    ]
    \addplot+[smooth] table[y=GMPS, x=NUM_WORK_GROUPS] {../data/vector-mult-add-group-1.dat};
    \addplot+[smooth] table[y=GMPS, x=NUM_WORK_GROUPS] {../data/vector-mult-add-group-2.dat};
    \addplot+[smooth] table[y=GMPS, x=NUM_WORK_GROUPS] {../data/vector-mult-add-group-4.dat};
    \addplot+[smooth] table[y=GMPS, x=NUM_WORK_GROUPS] {../data/vector-mult-add-group-8.dat};
    %\addplot+[smooth] table[y=GMPS, x=NUM_WORK_GROUPS] {../data/vector-mult-add-group-16.dat};
    \addplot+[smooth] table[y=GMPS, x=NUM_WORK_GROUPS] {../data/vector-mult-add-group-32.dat};
    \addplot+[smooth] table[y=GMPS, x=NUM_WORK_GROUPS] {../data/vector-mult-add-group-64.dat};
    \addplot+[smooth] table[y=GMPS, x=NUM_WORK_GROUPS] {../data/vector-mult-add-group-128.dat};
    \addplot+[smooth] table[y=GMPS, x=NUM_WORK_GROUPS] {../data/vector-mult-add-group-256.dat};
    \addplot+[smooth] table[y=GMPS, x=NUM_WORK_GROUPS] {../data/vector-mult-add-group-512.dat};
    \addplot+[smooth] table[y=GMPS, x=NUM_WORK_GROUPS] {../data/vector-mult-add-group-1024.dat};
    \legend{1, 2, 4, 8, 32, 64, 128, 256, 512, 1024}
    \addlegendimage{empty legend}
    \addlegendentry{Global Size \hfill}
    \end{loglogaxis}
\end{tikzpicture}

Both multiply and multiply-add show the same performance changes as the
number of elements increases. This shows that the addition of an extra
operation (no pun intended) does not affect the GPU performance.

As the global work size increases (as the number of elements increases)
the performance continues to improve. This tells us that performance is
(again) independent of the local work group size. 

When the number of elements reaches $32768$ there is a noticeable drop in
performance followed by a continual climb across the board. This is most
likely due to maxing out the 384 CUDA cores available on the GPU causing
the overhead of swapping front/back buffers to be added.

\section*{Tables: Multiply (Local)}
\lstinputlisting{../data/vector-mult-1.dat}
\lstinputlisting{../data/vector-mult-2.dat}
\lstinputlisting{../data/vector-mult-3.dat}
\lstinputlisting{../data/vector-mult-4.dat}
\lstinputlisting{../data/vector-mult-5.dat}
\lstinputlisting{../data/vector-mult-6.dat}
\lstinputlisting{../data/vector-mult-7.dat}
\lstinputlisting{../data/vector-mult-8.dat}
\lstinputlisting{../data/vector-mult-9.dat}
\lstinputlisting{../data/vector-mult-10.dat}
\lstinputlisting{../data/vector-mult-11.dat}
\lstinputlisting{../data/vector-mult-12.dat}
\lstinputlisting{../data/vector-mult-13.dat}
\lstinputlisting{../data/vector-mult-14.dat}

\section*{Tables: Multiply-Add (Local)}
\lstinputlisting{../data/vector-mult-add-1.dat}
\lstinputlisting{../data/vector-mult-add-2.dat}
\lstinputlisting{../data/vector-mult-add-3.dat}
\lstinputlisting{../data/vector-mult-add-4.dat}
\lstinputlisting{../data/vector-mult-add-5.dat}
\lstinputlisting{../data/vector-mult-add-6.dat}
\lstinputlisting{../data/vector-mult-add-7.dat}
\lstinputlisting{../data/vector-mult-add-8.dat}
\lstinputlisting{../data/vector-mult-add-9.dat}
\lstinputlisting{../data/vector-mult-add-10.dat}
\lstinputlisting{../data/vector-mult-add-11.dat}
\lstinputlisting{../data/vector-mult-add-12.dat}
\lstinputlisting{../data/vector-mult-add-13.dat}
\lstinputlisting{../data/vector-mult-add-14.dat}

\section*{Tables: Multiply (Global)}
\lstinputlisting{../data/vector-mult-group-1.dat}
\lstinputlisting{../data/vector-mult-group-2.dat}
\lstinputlisting{../data/vector-mult-group-4.dat}
\lstinputlisting{../data/vector-mult-group-8.dat}
\lstinputlisting{../data/vector-mult-group-32.dat}
\lstinputlisting{../data/vector-mult-group-64.dat}
\lstinputlisting{../data/vector-mult-group-128.dat}
\lstinputlisting{../data/vector-mult-group-256.dat}
\lstinputlisting{../data/vector-mult-group-512.dat}
\lstinputlisting{../data/vector-mult-group-1024.dat}

\section*{Tables: Multiply-Add (Global)}
\lstinputlisting{../data/vector-mult-add-group-1.dat}
\lstinputlisting{../data/vector-mult-add-group-2.dat}
\lstinputlisting{../data/vector-mult-add-group-4.dat}
\lstinputlisting{../data/vector-mult-add-group-8.dat}
\lstinputlisting{../data/vector-mult-add-group-32.dat}
\lstinputlisting{../data/vector-mult-add-group-64.dat}
\lstinputlisting{../data/vector-mult-add-group-128.dat}
\lstinputlisting{../data/vector-mult-add-group-256.dat}
\lstinputlisting{../data/vector-mult-add-group-512.dat}
\lstinputlisting{../data/vector-mult-add-group-1024.dat}

\end{document}
