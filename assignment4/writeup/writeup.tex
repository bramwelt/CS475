\documentclass[12pt]{article}

\usepackage{amsmath, amssymb, amsthm}
\usepackage{enumerate}
\usepackage{multicol}
\usepackage{listings}
\usepackage{listings}
\usepackage{changepage}
\usepackage{tikz}
\usepackage{pgfplots, pgfplotstable}
\usepackage{caption}

\usepgfplotslibrary{dateplot}
\usepgfplotslibrary{colormaps}

\lstset{
    basicstyle=\tiny\ttfamily
}

% Remove paragraph indents
\setlength{\parindent}{0in}

\title{CS475: Project 4}
\author{Trevor Bramwell}
\date{\today}

\begin{document}
\maketitle

\section*{Functional Decomposition}

Functional decomposition involves breaking an application down into it's
functional parts. For this project OpenMP was used to break down a
grain-growing simulation. The purpose of this project was to take
advantage of OpenMP's section and task directives, while learning about
barriers.\\

I added my own flair to the simulation by having an occasional
brush-fire wipe out 20\% of the grain. This was done by adding another
thread which set the NowFire variable, which in turn was used by the
new grain height computation function. The effects of this change can be
seen in the graph below.\\

The program was compiled using \emph{gcc -fopenmp -lm -O3} and ran on my
personal laptop: a Lenovo ThinkPad W540 (8 CPUs - 4 cores per socket, 2
threads per core, and 1 socket).\\

My program used the following combination of parameters:

\begin{itemize}
    \item Simulation Variables
        \begin{multicols}{2}
            \begin{itemize}
                \item Temperature
                \item Precipitation
                \item Grain
                \item Fire
                \item Deer
            \end{itemize}
        \end{multicols}
\end{itemize}


\section*{Graph}

\pgfplotsset{width=5in}
\pgfplotstableread{../data/graindeer.dat}\graindeer

\begin{tikzpicture}
    \begin{axis}[
        ymin=-10,
        title=Growth vs. Time,
        xlabel=Time,
        ylabel=Growth,
        legend style={
            legend pos=outer north east
        }
    ]
    \addplot+[smooth, mark=none] table[y=TEMP(C), x=STEP] {\graindeer};
    \addplot+[smooth, mark=none] table[y=PRECIP(cm), x=STEP] {\graindeer};
    \addplot+[smooth, mark=none,color=green!60!black] table[y=GRAIN(cm), x=STEP] {\graindeer};
    \addplot+[smooth, mark=none] table[y=DEER, x=STEP] {\graindeer};
    \addplot+[smooth, mark=none, color=orange, thick] table[y=FIRE, x=STEP] {\graindeer};
    \legend{Temperature, Precipitation, Grain, Deer, Fire}
    \addlegendimage{empty legend}
    \addlegendentry{Simultation \hfill}
    \end{axis}
\end{tikzpicture}

\section*{Patterns}

As the grain increased, more deer were born and ate more. Each time a
wildfire broke out the grain was reduced significantly, and as a result
the deer started dying.

\section*{Tables}

\begin{figure*}[b]
\lstinputlisting{../data/graindeer.dat}
\captionof{table}{Grain Deer Simulation}
\end{figure*}


\end{document}
