\documentclass[12pt]{article}

\usepackage{amsmath, amssymb, amsthm}
\usepackage{enumerate}
\usepackage{multicol}
\usepackage{listings}
\usepackage{changepage}
\usepackage{tikz}
\usepackage{pgfplots, pgfplotstable}
\usepackage{caption}

% Remove paragraph indents
\setlength{\parindent}{0in}

\title{CS475: Project 2}
\author{Trevor Bramwell}
\date{\today}

\begin{document}
\maketitle

\section*{N-Bodies}

This project involved comparing the parallel performance of
corse-grained, fine-grained, static scheduling, and dynamic scheduling
for running an N-body simulation using the OpenMP library.\\

The program was compiled using \emph{gcc} and ran on the
\emph{rabbit.engr.oregonstate.edu} Xeon processor with 32 CPUs (not the
Phi). It used the following combination of parameters:

\begin{itemize}
    \item Number of Bodies: 1024
    \item Number of Steps: 512
    \item Threads:
\begin{multicols}{2}
\begin{itemize}
    \item 1
    \item 2
    \item 4
    \item 8
    \item 16
    \item 31
    \item 32
\end{itemize}
\end{multicols}
\end{itemize}


\section*{Graph}

\pgfplotsset{width=5in}

\begin{tikzpicture}
    \begin{semilogxaxis}[
        ymin=0,
        xtick={1,2,4,8,16,32},
        title=Mega Bodies Computed Per Second(MBCPs) vs. Threads,
        xlabel=Threads,
        ylabel=MBCPs,
        log basis x={2},
        legend style={
            legend pos=outer north east
        }
    ]
    \addplot+[smooth] table[x=NUMT,y=MBCPs] {../data/corse-static.dat};
    \addplot+[smooth] table[x=NUMT,y=MBCPs] {../data/corse-dynamic.dat};
    \addplot+[smooth] table[x=NUMT,y=MBCPs] {../data/fine-static.dat};
    \addplot+[smooth] table[x=NUMT,y=MBCPs] {../data/fine-dynamic.dat};
    \legend{Corse-Static, Corse-Dynamic, Fine-Static, Fine-Dynamic}
    \addlegendimage{empty legend}
    \addlegendentry{Parallelism \hfill}
    \end{semilogxaxis}
\end{tikzpicture}

\section*{Patterns}

Between Corse and Fine grained parallelism, the data show corse
peforming much better. This is most likely do to the overhead of
assigning too little work to each thread. For fine-grained parallelism,
instead of working on an entire row of bodies, each thread only works on
one at a time.\\

The data shows static scheduling performs much better than dynamic for
both corse and fine grained parallelism.\\

Performance increases tremendously between 8 and 16 threads, and I
believe this is because the program starts taking advantage of having
full memory cache lines.\\

Sadly, there was not enough time to finish computation for the 32
threads. I have a strong feeling this means that performance at 32
threads or greater is abysmal.

\section*{Tables}

\begin{figure}[h]
\pgfplotstabletypeset[string type]{../data/corse-static.dat}
\captionof{table}{Corse-Static}
\end{figure}

\begin{figure}[h]
\pgfplotstabletypeset[string type]{../data/corse-dynamic.dat}
\captionof{table}{Corse-Dynamic}
\end{figure}
\begin{figure}[h]
\pgfplotstabletypeset[string type]{../data/fine-static.dat}
\captionof{table}{Fine-Static}
\end{figure}

\begin{figure}[h]
\pgfplotstabletypeset[string type]{../data/fine-dynamic.dat}
\captionof{table}{Fine-Dynamic}
\end{figure}


\end{document}
